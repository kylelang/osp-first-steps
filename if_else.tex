\HeaderA{if\_else}{Vectorised if-else}{if.Rul.else}
%
\begin{Description}
\code{if\_else()} is a vectorized \LinkA{if-else}{if}. Compared to the base R equivalent,
\code{\LinkA{ifelse()}{ifelse}}, this function allows you to handle missing values in the
\code{condition} with \code{missing} and always takes \code{true}, \code{false}, and \code{missing}
into account when determining what the output type should be.
\end{Description}
%
\begin{Usage}
\begin{verbatim}
if_else(condition, true, false, missing = NULL, ..., ptype = NULL, size = NULL)
\end{verbatim}
\end{Usage}
%
\begin{Arguments}
\begin{ldescription}
\item[\code{condition}] A logical vector

\item[\code{true}, \code{false}] Vectors to use for \code{TRUE} and \code{FALSE} values of
\code{condition}.

Both \code{true} and \code{false} will be \LinkA{recycled}{recycled}
to the size of \code{condition}.

\code{true}, \code{false}, and \code{missing} (if used) will be cast to their common type.

\item[\code{missing}] If not \code{NULL}, will be used as the value for \code{NA} values of
\code{condition}. Follows the same size and type rules as \code{true} and \code{false}.

\item[\code{...}] These dots are for future extensions and must be empty.

\item[\code{ptype}] An optional prototype declaring the desired output type. If
supplied, this overrides the common type of \code{true}, \code{false}, and \code{missing}.

\item[\code{size}] An optional size declaring the desired output size. If supplied,
this overrides the size of \code{condition}.
\end{ldescription}
\end{Arguments}
%
\begin{Value}
A vector with the same size as \code{condition} and the same type as the common
type of \code{true}, \code{false}, and \code{missing}.

Where \code{condition} is \code{TRUE}, the matching values from \code{true}, where it is
\code{FALSE}, the matching values from \code{false}, and where it is \code{NA}, the matching
values from \code{missing}, if provided, otherwise a missing value will be used.
\end{Value}
%
\begin{Examples}
\begin{ExampleCode}
x <- c(-5:5, NA)
if_else(x < 0, NA, x)

# Explicitly handle `NA` values in the `condition` with `missing`
if_else(x < 0, "negative", "positive", missing = "missing")

# Unlike `ifelse()`, `if_else()` preserves types
x <- factor(sample(letters[1:5], 10, replace = TRUE))
ifelse(x %in% c("a", "b", "c"), x, NA)
if_else(x %in% c("a", "b", "c"), x, NA)

# `if_else()` is often useful for creating new columns inside of `mutate()`
starwars %>%
  mutate(category = if_else(height < 100, "short", "tall"), .keep = "used")
\end{ExampleCode}
\end{Examples}
